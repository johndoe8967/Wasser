\documentclass[a4paper,12pt]{report}
\usepackage{amsmath} % Advanced math typesetting
\usepackage[utf8]{inputenc} % Unicode support (Umlauts etc.)
\usepackage[german]{babel} % Change hyphenation rules
\usepackage{hyperref} % Add a link to your document
\usepackage{graphicx} % Add pictures to your document
\usepackage{listings} % Source code formatting and highlighting
\usepackage[utf8]{inputenc}
\begin{document}\selectlanguage{german}
\title{Wasserwächter}
\author{Michael Petruzelka\\
\texttt{michael.petruzelka@liwest.at}}
\date{\today}
\maketitle
\begin{abstract}
Kleine Leckagen im Rohrleitungssystem eines Hauses können neben dem erhöhten Wasserverbrauch auch große Schäden 
an der Gebäudesubstanz und den Einrichtungsgegenständen anrichten. 

Ziel dieses Projektes ist es Leckagen zu entdecken und zu melden bevor es zu nennenswerten Schäden kommt. 
Dazu ist es notwendig den beabsichtigten Gebrauch von Wasser vom unbeabsichtigten Auslaufen zu unterscheiden.

Als erstes muss der Wasserverbrauch detektiert werden, 
dazu ist es zweck-mäßig den bereits vorhandenen Wasserzähler heranzuziehen. 
Typischerweise haben diese ein präzises Zählwerk, welches jedoch nur optisch abgetastet werden kann.

Die Schwierigkeit besteht nun darin geeignete Algorithmen zu finden die zuverlässig erkennen aber keine Fehlalarme produzieren
\end{abstract}
\tableofcontents
\chapter{Messung}
Hallo das ist ein Test
Ich mag ja Latex nicht wirklich
\chapter{Verarbeitung}
\chapter{Speicherung}
\chapter{Source}

\end{document}